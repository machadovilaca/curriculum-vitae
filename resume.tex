\documentclass{resume}

%encoding
%--------------------------------------
\usepackage[T1]{fontenc}
\usepackage[utf8]{inputenc}
%--------------------------------------

%Portuguese-specific commands
%--------------------------------------
\usepackage[portuguese, english]{babel}
%--------------------------------------

%Hyphenation rules
%--------------------------------------
\usepackage{hyphenat}
\hyphenation{mate-mática recu-perar}

\usepackage{ragged2e}
\usepackage{fontawesome}
\usepackage[left=0.5in,top=0.2in,right=0.5in,bottom=0.6in]{geometry} % Document margins
\usepackage{enumitem}

\usepackage{fancyhdr}
\pagestyle{fancy}
\renewcommand{\headrulewidth}{0pt}

\newcommand{\tab}[1]{\hspace{.2667\textwidth}\rlap{#1}}
\newcommand{\itab}[1]{\hspace{0em}\rlap{#1}}

\name{João Vilaça}
\address{Software Engineer}
\address{1998-06-30 \\ Braga, Portugal}

\cfoot{Updated on {\selectlanguage{english}\today}}

\begin{document}
\begin{center}
    \href{mailto:machadovilaca@gmail.com}{machadovilaca@gmail.com} \hspace{0.1cm}
    \\
    Find me as \textit{machadovilaca} on all Social Media:
    \href{https://www.linkedin.com/in/machadovilaca}{\faLinkedin} \hspace{0.1cm}
    \href{https://github.com/machadovilaca}{\faGithub} \hspace{0.1cm}
    \href{https://gitlab.com/machadovilaca}{\faGitlab} \hspace{0.1cm}
\end{center}

%----------------------------------------------------------------------------------------
%	WORK EXPERIENCE SECTION
%----------------------------------------------------------------------------------------

\begin{rSection}{Work Experience}

\begin{rSubsection}
{\textbf{Software Engineer}, Red Hat \href{https://www.redhat.com}{\faLink}}{2021 - Present}{}{}
Working in the \href{https://github.com/kubevirt}{KubeVirt} project, a tool for managing virtualization workloads on Kubernetes. I'm mainly contributing to \href{https://github.com/kubevirt/hyperconverged-cluster-operator}{Hyperconverged Cluster Operator}, a unified Kubernetes operator for deploying and controlling several adjacent operators.
\end{rSubsection}

\begin{rSubsection}
{\textbf{DevOps Engineer}, Public Mint \href{https://publicmint.com/}{\faLink}}{2019 - 2021}{}{}
Kubernetes orchestration, logging, monitoring and tracing. AWS infrastructure management and automation with Terraform and Ansible. CI/CD with GitLab
\end{rSubsection}

\begin{rSubsection}
{\textbf{2 DevOps Summer Internships}, Eurotux Informática \href{https://eurotux.com/}{\faLink}}{2017 and 2018}{}{}
Open Source contributions for OPNsense Firewall, Nginx and Nginx Cache Purge Module. Go service for access in development environments to Docker containers via SSH running in any EC2 machine on a given ECS service
\end{rSubsection}

\end{rSection}

%----------------------------------------------------------------------------------------
%	EDUCATION SECTION
%----------------------------------------------------------------------------------------

\begin{rSection}{Education}

\begin{rSubsection}
{\textbf{Student Researcher}, AIDA - INESC TEC \href{https://www.inesctec.pt/en/projects/aida}{\faLink}}{2020 - 2022}{}{}

Design and implementation of new scheduling frameworks for Edge Computing that accounts for both the available resources and the geographical location of nodes when orchestrating workloads in Kubernetes/KubeEdge

\end{rSubsection}

\begin{rSubsection}
{\textbf{Integrated Master Degree in Software Engineering}, University of Minho \href{https://www.uminho.pt/EN}{\faLink}}{2016 - 2022}{}{}
% \textit{Thesis:} Orchestration and distribution of services in hybrid Cloud/Edge environments - Development of a Kubernetes/KubeEdge scheduler, aware of the geographic location of processing nodes and of data producers to improve in scalability levels, while reducing request latency and network usage
% \vspace{0.2cm}
% \\
\textit{Graduate Coursework:} Distributed Systems, Fault Tolerance, Large Scale Systems, Semantic Interoperability, Knowledge Discovery, Data Analysis, IT Systems Security
% \vspace{0.2cm}
% \\
% \textit{Undergraduate Coursework:} Programming Language Paradigms, Computer Architectures, Distributed Systems, Operating Systems, Algorithms, Project Analysis and Design, Databases, Networking, Mathematics \hspace{0.2cm}
% \vspace{0.2cm}
% \\
% \textit{Extra activities:} Class delegate, Member of the Pedagogic Council, Academic Newspaper \hspace{0.2cm}
\end{rSubsection}

\end{rSection}

%----------------------------------------------------------------------------------------
%	EDUCATION SECTION
%----------------------------------------------------------------------------------------

\begin{rSection}{Publications}

\begin{rSubsection}
{\textbf{Geolocate: A geolocation-aware scheduling system for Edge Computing}}{2021}{}{}
Vilaça J, Paulo J, Vilaça R. Geolocate: A geolocation-aware scheduling system for Edge Computing. Workshop on High-Performance and Reliable Big Data (HPBD), colocated with SRDS. Available \href{https://dsr-haslab.github.io/repository/vpv21.pdf}{here}.

\end{rSubsection}

\end{rSection}

%----------------------------------------------------------------------------------------
%	TECHNICAL STRENGTHS SECTION
%----------------------------------------------------------------------------------------

\begin{rSection}{Strengths}

\begin{tabular}{ @{} >{\bfseries}l @{\hspace{3ex}} l }
Languages and Technologies \ & Go, Java, Python, Ruby (Rails), Haskell, Elixir (Phoenix), Kubernetes, \\
\vspace{0.1cm}
    \ & Openshift, AWS, Docker, Terraform, Ansible, CI/CD, ELK, Prometheus \\
Personal \ & Critical and Logical Thinking, Self-criticism, Leadership, Communication, \\
    \ & Perseverance, Writing Experience
\end{tabular}

\end{rSection}

%----------------------------------------------------------------------------------------
%	VOLUNTEER EXPERIENCE SECTION
%----------------------------------------------------------------------------------------

\begin{rSection}{Volunteer Experience}

\begin{rSubsection}
{\textbf{CeSIUM - University of Minho Computer Engineering Students Center}}{}{}{}
\begin{shifted}
2020/2021 - Vice-President
\\
2019/2020 - Treasurer
\\
2018/2019 - Secretary (board of administration)
\\
2017/2018 - Department of Communication Co-director
\end{shifted}
\end{rSubsection}

\begin{rSubsection}
{\textbf{ENEI - Portuguese National Meeting of Computer Science Students}}{}{}{}
\begin{shifted}
2019/2021 - Treasurer
\end{shifted}
\end{rSubsection}

\end{rSection}

\end{document}
